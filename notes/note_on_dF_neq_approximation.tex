\documentclass[11pt]{article}
\usepackage[margin=1in]{geometry}                % See geometry.pdf to learn the layout options. There are lots.
\usepackage{url}
\geometry{letterpaper}                   % ... or a4paper or a5paper or ... 
\usepackage[parfill]{parskip}    % Activate to begin paragraphs with an empty line rather than an indent
\usepackage{graphicx}
\usepackage{amsmath}
\usepackage{amssymb}
\usepackage{epstopdf}
\usepackage{microtype}
\usepackage{heuristica}
\usepackage{cancel}
\usepackage{float}

\newcommand{\x}{\mathbf{x}}
\newcommand{\s}{\mathbf{s}}
\newcommand{\vel}{\mathbf{v}}
\newcommand{\kldiv}{\mathcal{D}_\text{KL}}

\begin{document}
\subsection*{Question}
Is this a valid near-equilibrium approximation for the free energy difference between $\omega$ and $\pi$?
$$F_\omega - F_\pi \approx\frac{1}{2} \left[
\langle W_\text{shad} \rangle_{\pi \to \rho}
- \langle W_\text{shad} \rangle_{\omega \to \rho} \right]$$


\subsection*{Notation}
\begin{itemize}
\item the equilibrium distribution is $\pi$
$$\pi(\x, \vel) \equiv p(\x) q(\vel)$$
\item the nonequilibrium steady state sampled by our Langevin integrator is $\rho$
$$\rho(\x, \vel) = \; ?$$
\item the configuration-marginal of the nonequilibrium steady state is $\rho_\x$
$$\rho_\x(\x) \equiv \int d\vel \rho(\x, \vel)$$
\item the ``midpoint'' nonequilibrium macrostate is $\omega$
$$\omega(\x, \vel) \equiv \rho_\x (\x) q(\vel)$$
\end{itemize}

\subsection*{Configuration space bias is equivalent to $F_\omega - F_\pi$}
We're interested in approximating the KL divergence between the configuration distribution at equilibrium and the configuration distribution in the nonequilibrium steady state: $\kldiv(\rho_\x \| p)$.
We've constructed $\omega$ so that $\kldiv(\rho_\x \| p) = \kldiv( \omega \| \pi)$:
$$\begin{aligned}
\kldiv(\omega \| \pi) &\equiv \int \int d \x d\vel \left( \omega(\x, \vel) \log \frac{\omega(\x, \vel)}{\pi(\x, \vel)} \right)\\
&= \int \int d \x d\vel \left( \omega(\x, \vel) \log \frac{\rho_\x (\x)}{p(\x)}  \cancelto{1}{\frac{q(\vel)}{q(\vel)}}\right)\\
&= \int \int d \x d\vel \left( \omega(\x, \vel) \log \frac{\rho_\x (\x)}{p(\x)}  \right)\\
&= \int d \vel \left( \int d \x \rho_\x (\x) q(\vel) \log \frac{\rho_\x (\x)}{p(\x)} \right) \\
&= \cancelto{1}{\left( \int d \vel q(\vel) \right)} \left( \int d \x \rho_\x (\x) \log \frac{\rho_\x (\x)}{p(\x)} \right) \\
&= \kldiv(\rho_\x \| p)
\end{aligned}
$$

%Thus, if we can estimate $\Delta F_\text{neq}$ between $\pi$ and $\omega$, 

%Since nonequilibrium free energy differences are equal to KL divergences, 
Note that $\kldiv(\omega \| \pi) = \beta (F_\omega - F_\pi)$ where $\beta = \frac{1}{k_B T}$ (see proof proceeding eq. 1 of \url{http://threeplusone.com/Sivak2012a.pdf}).

Thus, if we can approximate the nonequilibrium free energy difference $F_\omega - F_\pi$, then we can approximate the configuration-space bias.

\subsection*{Approximating $F_\omega - F_\pi$}
We can write down a nonequilibrium protocol that transforms $\pi$ into $\omega$:
\begin{enumerate}
\item \textbf{Simulate Langevin dynamics} for $N$ steps (performs ``shadow work'' to turn $\pi$ into $\rho$)
\item \textbf{Randomize velocities} by sampling $\vel \sim q$ (performs no work, turns $\rho$ into $\omega$)
\end{enumerate}
where $N$ is chosen sufficiently large to reach steady state $\rho$.

Call the procedure [1,2] the ``forward protocol'' and the procedure [2,1] the ``reverse protocol.''

Using the near-equilibrium approximation in \url{http://threeplusone.com/Sivak2012a.pdf}, we can approximate $\Delta F_\text{neq} \equiv F_\omega - F_\pi$  in terms of quantities we can compute from simulation.

Denote by $\langle W_\text{F} \rangle_\pi$ the average work performed by the ``forward protocol,'' where the average is over initial conditions at equilibrium. Denote by $\langle W_\text{R} \rangle_\omega$ the average work performed by the ``reverse protocol,'' where the average is over initial conditions distributed according to $\omega$.

$$\begin{aligned}
\Delta F_\text{neq}
& \approx
\frac{1}{2} [ \langle W_\text{F} \rangle_\pi - \langle W_\text{R} \rangle_\omega ]\\
&= \frac{1}{2}
\left[
\left(
\langle W \rangle_{\pi \to \rho}
+ \langle W \rangle_{\rho \to \omega}
\right)
- \left(
\langle W \rangle_{\omega \to \omega} + 
\langle W \rangle_{\omega \to \rho}
\right)
\right]\\
&= \frac{1}{2} [
\langle W_\text{shad} \rangle_{\pi \to \rho}
+ \cancelto{0}{\langle W \rangle_{\rho \to \omega}}
- 
\cancelto{0}{\langle W \rangle_{\omega \to \omega}} 
- 
\langle W_\text{shad} \rangle_{\omega \to \rho} ]\\
&= \frac{1}{2} \left[
\langle W_\text{shad} \rangle_{\pi \to \rho}
- \langle W_\text{shad} \rangle_{\omega \to \rho} \right]\\
\end{aligned}$$

Where we can approximate:
\begin{itemize}
\item $\langle W_\text{shad} \rangle_{\pi \to \rho}$ by starting in equilibrium, and measuring the shadow work accumulated by simulating for $N$ steps.
\item $\langle W_\text{shad} \rangle_{\omega \to \rho}$ by starting in nonequilibrium steady state with velocities resampled from equilibrium, and measuring the shadow work accumulated by simulating for $N$ steps.
\end{itemize}
\end{document}